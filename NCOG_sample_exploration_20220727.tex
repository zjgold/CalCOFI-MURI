% Options for packages loaded elsewhere
\PassOptionsToPackage{unicode}{hyperref}
\PassOptionsToPackage{hyphens}{url}
%
\documentclass[
]{article}
\usepackage{amsmath,amssymb}
\usepackage{lmodern}
\usepackage{iftex}
\ifPDFTeX
  \usepackage[T1]{fontenc}
  \usepackage[utf8]{inputenc}
  \usepackage{textcomp} % provide euro and other symbols
\else % if luatex or xetex
  \usepackage{unicode-math}
  \defaultfontfeatures{Scale=MatchLowercase}
  \defaultfontfeatures[\rmfamily]{Ligatures=TeX,Scale=1}
\fi
% Use upquote if available, for straight quotes in verbatim environments
\IfFileExists{upquote.sty}{\usepackage{upquote}}{}
\IfFileExists{microtype.sty}{% use microtype if available
  \usepackage[]{microtype}
  \UseMicrotypeSet[protrusion]{basicmath} % disable protrusion for tt fonts
}{}
\makeatletter
\@ifundefined{KOMAClassName}{% if non-KOMA class
  \IfFileExists{parskip.sty}{%
    \usepackage{parskip}
  }{% else
    \setlength{\parindent}{0pt}
    \setlength{\parskip}{6pt plus 2pt minus 1pt}}
}{% if KOMA class
  \KOMAoptions{parskip=half}}
\makeatother
\usepackage{xcolor}
\usepackage[margin=1in]{geometry}
\usepackage{longtable,booktabs,array}
\usepackage{calc} % for calculating minipage widths
% Correct order of tables after \paragraph or \subparagraph
\usepackage{etoolbox}
\makeatletter
\patchcmd\longtable{\par}{\if@noskipsec\mbox{}\fi\par}{}{}
\makeatother
% Allow footnotes in longtable head/foot
\IfFileExists{footnotehyper.sty}{\usepackage{footnotehyper}}{\usepackage{footnote}}
\makesavenoteenv{longtable}
\usepackage{graphicx}
\makeatletter
\def\maxwidth{\ifdim\Gin@nat@width>\linewidth\linewidth\else\Gin@nat@width\fi}
\def\maxheight{\ifdim\Gin@nat@height>\textheight\textheight\else\Gin@nat@height\fi}
\makeatother
% Scale images if necessary, so that they will not overflow the page
% margins by default, and it is still possible to overwrite the defaults
% using explicit options in \includegraphics[width, height, ...]{}
\setkeys{Gin}{width=\maxwidth,height=\maxheight,keepaspectratio}
% Set default figure placement to htbp
\makeatletter
\def\fps@figure{htbp}
\makeatother
\setlength{\emergencystretch}{3em} % prevent overfull lines
\providecommand{\tightlist}{%
  \setlength{\itemsep}{0pt}\setlength{\parskip}{0pt}}
\setcounter{secnumdepth}{-\maxdimen} % remove section numbering
\ifLuaTeX
  \usepackage{selnolig}  % disable illegal ligatures
\fi
\IfFileExists{bookmark.sty}{\usepackage{bookmark}}{\usepackage{hyperref}}
\IfFileExists{xurl.sty}{\usepackage{xurl}}{} % add URL line breaks if available
\urlstyle{same} % disable monospaced font for URLs
\hypersetup{
  pdftitle={NCOG eDNA Sample Exploration},
  hidelinks,
  pdfcreator={LaTeX via pandoc}}

\title{NCOG eDNA Sample Exploration}
\author{}
\date{\vspace{-2.5em}2022-07-26}

\begin{document}
\maketitle

\hypertarget{tldr}{%
\section{TL;DR}\label{tldr}}

\hypertarget{ncog-has-tons-of-surface-samples-a-handful-of-cardinal-stations-are-more-frequently-sampled-and-most-stations-have-at-least-5-samples.}{%
\subsubsection{NCOG has tons of surface samples, a handful of cardinal
stations are more frequently sampled, and most stations have at least 5
samples.}\label{ncog-has-tons-of-surface-samples-a-handful-of-cardinal-stations-are-more-frequently-sampled-and-most-stations-have-at-least-5-samples.}}

\hypertarget{summary-of-ncog}{%
\section{Summary of NCOG}\label{summary-of-ncog}}

NOAA-CalCOFI Ocean Genomics samples have been conducted regularly on
CalCOFI cruises from 2014 until the present lead by Professor Andrew
Allen and his lab at Scripps Institution of Oceanography + J. Craig
Venter Institute. These samples are seawater collected at sea via Niskin
rosettes on the CalCOFI cruises. {[}Not to be confused with ethanol
preserved samples of 505 µm plankton tows that Zack amplified during his
SWFSC internship.{]}

Previous work utilizing the NCOG samples have focussed on characterizing
microbial and phytoplankton diversity. Here is their most recent
publication which includes all currently extracted samples:
\url{https://www.nature.com/articles/s41467-022-30139-4.pdf?origin=ppub)=}

The purpose of this document is to summarize the currently available
NCOG samples to 1) understand the distribution of samples across space,
time, and depth, and 2) help identify priority targets for the MURI
CalCOFI sampling.

\hypertarget{protocols}{%
\subsubsection{Protocols}\label{protocols}}

\textbf{Collection Protocol}:
\url{https://www.protocols.io/view/noaa-calcofi-ocean-genomics-ncog-sample-collection-eq2lypdorlx9/v1}\\
For DNA samples, 0.5-2L liters of seawater collected via Niskin are
filtered via Masterflex persistaltic pumps onto 0.2 µm Sterivex filters.

\textbf{DNA Extraction Protocol}:
\url{https://www.protocols.io/view/sterivex-dna-extraction-x54v9m1y4g3e/v2}\\
DNA is extracted via NucleoMag Plant Kit for DNA purification
(Macherey-Nagel, Düren, Germany) on an epMotion 5057TMX (Eppendorf,
Hamburg, Germany).

\textbf{Library Preparation
Protocol}:\url{https://www.protocols.io/view/amplicon-library-preparation-bp2l6b4j5gqe/v1}\\
V4-V5 region of the 16 S rRNA gene and V9 region of the 18S rRNA
One-step PCR using the TruFi DNA Polymerase PCR kit

\hypertarget{general-statistics}{%
\section{General Statistics}\label{general-statistics}}

Unique Samples:

\begin{verbatim}
## [1] "1491"
\end{verbatim}

Unique Samples by Depth:

\begin{longtable}[]{@{}lr@{}}
\toprule()
Depth & Count \\
\midrule()
\endhead
{[}-5,5{]} & 11 \\
(5,15{]} & 752 \\
(15,25{]} & 102 \\
(25,35{]} & 121 \\
(35,45{]} & 116 \\
(45,55{]} & 94 \\
(55,65{]} & 68 \\
(65,75{]} & 54 \\
(75,85{]} & 39 \\
(85,95{]} & 65 \\
(95,105{]} & 28 \\
(105,115{]} & 23 \\
(115,125{]} & 4 \\
(125,135{]} & 2 \\
(165,175{]} & 5 \\
(505,515{]} & 7 \\
\bottomrule()
\end{longtable}

\hypertarget{histogram-of-depth}{%
\subsection{Histogram of Depth}\label{histogram-of-depth}}

\includegraphics{NCOG_sample_exploration_20220727_files/figure-latex/unnamed-chunk-5-1.pdf}

Clearly the vast majority of NCOG samples are taken within the top 100m.
Only 7 samples were taken below 200m.

\hypertarget{maps}{%
\section{Maps}\label{maps}}

\hypertarget{cruise-map}{%
\subsection{Cruise Map}\label{cruise-map}}

\includegraphics{NCOG_sample_exploration_20220727_files/figure-latex/unnamed-chunk-6-1.pdf}

The cardinal stations are visible here as they are sampled far more
frequently.

\hypertarget{bottle-map}{%
\subsection{Bottle Map}\label{bottle-map}}

\includegraphics{NCOG_sample_exploration_20220727_files/figure-latex/unnamed-chunk-7-1.pdf}

The vast majority of stations south of Point. Conception have multiple
depths and multiple cruises.

\hypertarget{deep-bottle-map}{%
\subsection{Deep Bottle Map}\label{deep-bottle-map}}

\includegraphics{NCOG_sample_exploration_20220727_files/figure-latex/unnamed-chunk-8-1.pdf}

Very few samples taken at depth.

\hypertarget{average-depth-map}{%
\subsection{Average Depth Map}\label{average-depth-map}}

\includegraphics{NCOG_sample_exploration_20220727_files/figure-latex/unnamed-chunk-9-1.pdf}

Another way of visualizing the depth distributions of the samples. One
of the sets of samples is taken at the Chla max which explains the
nearshore-offshore depth distribution. Cardinal stations have deeper
samples taken.

\hypertarget{tile-plot}{%
\section{Tile Plot}\label{tile-plot}}

\includegraphics{NCOG_sample_exploration_20220727_files/figure-latex/unnamed-chunk-10-1.pdf}

Only a handful of stations have been continously sampled on nearly every
cruise since 2014. The vast majority of stations have been sampled more
infrequently. Northern stations were sampled 2x times.

\end{document}
